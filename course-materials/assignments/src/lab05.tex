\documentclass[12pt]{article}
\usepackage{listings}
\usepackage{color}
\textwidth=7in
\textheight=9.5in
\topmargin=-1in
\headheight=0in
\headsep=.5in
\hoffset  -.85in

\definecolor{mygray}{rgb}{0.4,0.4,0.4}
\definecolor{mygreen}{rgb}{0,0.8,0.6}
\definecolor{myorange}{rgb}{1.0,0.4,0}

\lstset{
basicstyle = \ttfamily,columns=fullflexible,
commentstyle=\color{mygray},
frame=single,
numbers=left,
numbersep=5pt,
numberstyle=\tiny\color{mygray},
keywordstyle=\color{mygreen},
showspaces=false,
showstringspaces=false,
stringstyle=\color{myorange},
tabsize=2
}

\pagestyle{empty}

\renewcommand{\thefootnote}{\fnsymbol{footnote}}

\begin{document}

\begin{center}
{\bf lab 05 Pointers}

\end{center}

\setlength{\unitlength}{1in}

\begin{picture}(6,.1) 
\put(0,0) {\line(1,0){6.25}}         
\end{picture}

\renewcommand{\arraystretch}{2}
\setlength{\tabcolsep}{6pt} % General space between cols (6pt standard)
\renewcommand{\arraystretch}{.5} % General space between rows (1 standard)

\vskip.15in
\noindent\textbf{Instructions:} It is \textbf{\textit{\underline{vital}}} that you understand pointers and, consequently, the different memory spaces: static, stack, and heap.  Please answer the following questions:

\vskip.15in
\noindent\textbf{Note:} When a question asks for the \textit{value} of a variable, if it is a known number write the number.  If it is a memory address write what the memory address is pointing to (Ex: The variable x holds the memory address of the variable y.)  If it cannot be determined, write undefined.


\begin{enumerate}
\item (15 pts) Consider the following code:\\
\begin{lstlisting}
#include <stdio.h>

int main(int argc, char *argv[]) {
    int x = 32;
    int *y = &x;
    int z = *y;
    printf("z = %d\n", z);
    return 0;
}

\end{lstlisting}
\begin{enumerate}
\item What is the output?
\item What is the \textit{value} of y?
\item In what memory (static, stack, or heap) do the variable x, y, z exist during runtime?
\end{enumerate}


\item (15 pts) Consider the following code:\\
\begin{lstlisting}
#include <stdio.h>

int main(int argc, char *argv[]) {
    int *x = (int *)200;
    long z = (long)x;
    printf("z = %d\n", z);
    return 0;
}

\end{lstlisting}
\begin{enumerate}
\item What is the output?
\item In what memory (static, stack, or heap) do the variable x, z exist during runtime?
\item Why is this code bad practice (even though it compiles/runs without a seg fault)?
\end{enumerate}

\pagebreak

\item (15 pts) Consider the following code:\\
\begin{lstlisting}
#include <stdio.h>

int main(int argc, char *argv[]) {
    int *x = new int[100];
    x[0] = 500;
    int z = x[25];
    printf("z = %d\n", z);
    return 0;
}
\end{lstlisting}
\begin{enumerate}
\item What is the output? (Careful, a compile/run will not give you the correct answer.)
\item In what memory (static, stack, or heap) do the variable x, z exist during runtime?
\item What is stored on the heap?
\item What is the memory issue?
\end{enumerate}

\item (15 pts) Consider the following code:\\
\begin{lstlisting}
#include <stdio.h>

int main(int argc, char *argv[]) {
    int *x = new int[100];
    int *y = x+10;

    for (int i = 0; i < 100; i++) {
        x[i] = i;
    }

    printf("y[10] = %d\n", y[10]);
    return 0;
}

\end{lstlisting}
\begin{enumerate}
\item Why is the output?: \\
y[10] = 20 \\
\item Write a line of code that would free up memory using only x.
\item Write a line of code that would free up memory using only y.
\end{enumerate}

\pagebreak

\item (15 pts) Consider the following code:\\
\begin{lstlisting}
#include <stdio.h>

int main(int argc, char *argv[]) {
    int *x = new int[100];
    int *y = new int[100];
    int **z = NULL;

    for (int i = 0; i < 100; i++) {
        x[i] = i;
        y[i] = 100-i;
    }

    z = &x;
    printf("(*z)[10] = %d\n", (*z)[10]);
    z = &y;
    printf("(*z)[10] = %d\n", (*z)[10]);
    return 0;
}
\end{lstlisting}
\begin{enumerate}
\item What is the \textit{value} of x and y?
\item What is the \textit{value} z on lines 14 and 16?
\item What does the code fragment ``(*z)[10]'' mean? (Describe what the code must do the evaluate that code fragment.)
\item Why are there two different outputs for ``(*z)[10]''?  Here is the output of the program:
(*z)[10] = 10 \\
(*z)[10] = 90
\item Write the commands to free memory.
\end{enumerate}

\pagebreak

\item (25 pts) Consider the following code:\\
\begin{lstlisting}
#include <stdio.h>
#include <stdlib.h>

class Student {
    public:
        int mId;
        double mGPA;
        std::string mAddress;
        std::string mBiography;
};

int main(int argc, char *argv[]) {
    Student *students = new Student[100];
    Student ** studentsPtr = new Student*[100];

    srand(100); // Seed random number generator
    for (int i = 0; i < 100; i++) {
        students[i].mId = i+1;

        // Generate a "random" GPA from 0.0-4.0
        students[i].mGPA = 4 * (((double)rand())/RAND_MAX);

        studentsPtr[i] = students+i;
    }

    // This is Bubble Sort:
    for (int i = 0; i < 100; i++) {
        for (int j = 1; j < 100; j++) {
            // Based on GPA
            if (studentsPtr[j-1]->mGPA > studentsPtr[j]->mGPA) {
                Student *temp = studentsPtr[j];
                studentsPtr[j] = studentsPtr[j-1];
                studentsPtr[j-1] = temp;
            }
        }
    }

    for (int i = 0; i < 100; i++) {
        printf("%f\n", studentsPtr[i]->mGPA);
    }

    return 0;
}

\end{lstlisting}
\begin{enumerate}
\item What is the \textit{value} of students?
\item What is the \textit{value} of studentsPtr?
\item Which of the above is being sorted?
\item Why was the above (in c) chosen over the other?  What advantage do you see?
\item Write the commands to free memory.
\end{enumerate}

\end{enumerate}

\vskip.15in
\noindent\textbf{How to turn in:} \\
Turn in via GitHub.  Ensure the file(s) are in your directory and then:
\begin{itemize}
\item \$ git add $<$files$>$
\item \$ git commit 
\item \$ git push
\end{itemize}

\vskip.15in
\noindent\textbf{Due Date:}
September 09, 2020 2359

\vskip.15in
\noindent\textbf{Teamwork:} No teamwork, your work must be your own.

\end{document}
