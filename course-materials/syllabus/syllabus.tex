\documentclass[10pt]{article}
\textwidth=7in
\textheight=9.5in
\topmargin=-1in
\headheight=0in
\headsep=.5in
\hoffset  -.85in

\pagestyle{empty}

\renewcommand{\thefootnote}{\fnsymbol{footnote}}

\usepackage{hyperref}

\begin{document}

\begin{center}
{\bf CSCI 315\ Sec. 01  Data Structures Analysis \ \ MW 800-0950 Lab 0950-1050,  Room: Ashby Hall 208
}
\end{center}

\setlength{\unitlength}{1in}

\begin{picture}(6,.1) 
\put(0,0) {\line(1,0){6.25}}         
\end{picture}

\renewcommand{\arraystretch}{2}
\setlength{\tabcolsep}{6pt} % General space between cols (6pt standard)
\renewcommand{\arraystretch}{.5} % General space between rows (1 standard)

\vskip.15in
\noindent\textbf{Instructor:} Dr. Paul E. West, AH 206, Phone: (843)-863-7329, email: pwest@csuniv.edu

\vskip.15in
\noindent\textbf{Discord:}\\ 
Course: \url{https://discord.gg/DcQw2kH} \\
Virtual Office: \url{https://discord.gg/bzn49gb} \\
ACM Club: \url{https://discord.gg/XsyUvFA}

\vskip.15in
\noindent\textbf{Office Hours:} M 1100-1200, TR 0740-0920, TR 1200-1350 and by appointment. If you use Discord, and you want a private conversation, please discord-call (right-click on me and select call.)

\vskip.15in
\noindent\textbf{Email:} Due to CSU's new email policy I will only guarantee email access during my office hours.

\vskip.15in
\noindent\textbf{Textbooks:} \\
\underline{C++ Programming: Program Design Including Data Structures} D. S. Malik. 9th Edition (whatever edition you used for CSCI 235 is also fine). You may use a different book if you like, as assignments and labs will come from class material and not the book.

\vskip.15in
\noindent\textbf{Github:} \\
I drive most of my classes from github.  Assignments and lectures are located on Github here: \url{https://github.com/csu-cs/csci-315-fall-2020}.  If you are not able to view, please send me an email so I can grant you access.\\
\textbf{Note:} The content for this course in on Github!

\vskip.15in
\noindent\textbf{Course Description:}
This course is designed to teach you many of the basic data structures used in computer science. You will learn how to make them, use them, and analyze them. In addition you will see how to apply them to novel problems using careful analysis of the problem to determine which data structure is best suited. Coding will be done in C++ though the concepts are not language dependent.

This course is one of the most important in the major. Projects will require a level of design that is new to your programming career and will help to prepare you for real-world jobs. The ‘toolbox’ this class will provide you will be used throughout your programming future. The course serves as a gateway to many advanced courses. CSCI 315 is a \textbf{demanding} course, expect more work than previous courses. The projects in this class are more involved, requiring a deeper understanding of Computer Science. Learning to design and manage large projects is a core concept of the degree.\\

\textbf{\textit{\underline{From previous experience, those who keep up with the work do well.}}}

\vskip.15in
\noindent\textbf{Teamwork:} \textbf{\textit{\underline{There are no group projects in CSCI 315.}}}

\vskip.15in
\noindent\textbf{Attendance and Late Work:} I will not enforce an attendance policy.  Any late work will receive a 20\% penalty and I will not accept \textit{late} work after November 17, 2020 at 2359.


\vskip.15in
\noindent\textbf{Tentative Schedule}: \\ \\
Section 1 Topics:
*nix topics: Bash, compilation, profiling, performance analysis

Section 2 Topics:
Arrays (review), abstract data types (classes), pointers, linked lists, object-oriented programming (OOP), strings. 

Section 3 Topics:
Inheritance, Big-O, Stacks, Queues, Templates, STL, casting, Sorting (insertion, selection, mergesort, quicksort), Streams. 

Section 4 Topics:
Recursion, Binary Search Trees (BST), Maps, Heaps, Priority Queues, Heapsort, Hash tables, Operator Overloads, Graphs.


\vskip.15in
\noindent\textbf{Grading}: \\ \\
\begin{tabular}{ l l}
Ethics Paper & 5\%\\
Midterm & 15\% \\
Final & 15\% \\
Labs & 35\% \\
Projects & 30\% \\
\end{tabular}\\

\vspace*{.15in}
\noindent\textbf{Grading Scale:} \\ \\
\begin{tabular}{|l|l|}
\hline
100 - 90 & A \\ \hline
89 - 87 & B+ \\ \hline
86 - 80 & B \\ \hline
79 - 77 & C+ \\ \hline
76 - 70 & C \\ \hline
69 - 60 & D \\ \hline
below 60 & F \\ \hline
\end{tabular}

\vskip.15in
\noindent\textbf{Weekends}:
I observe the Sabbath on Sunday. On Saturday I spend most of the day with my family.  Therefore, I shall not be able to communicate on Sunday and there is a low chance I will on Saturday.  Please keep this in mind when working on assignments--especially for the online students!

\vskip.15in
\noindent\textbf{Class Recording}:
I do record my classes via screen capture and camera.  After class (or during lab) I will upload and post a link on BlackBoard.  This is done both for online students and so you can review the material.  If you do not see the content within 24 hours of class, please email me a reminder.

\vskip.15in
\noindent\textbf{Online Students}:
I understand that some students must take this class online.  Be aware that the lack of face to face communication puts online students at a \textbf{great} disadvantage.  Therefore communication is paramount.  I will be able to answer your email during office hours.  I am not able (due to security reasons) to receive notifications of emails outside of my office.  Please keep the timing of due dates for labs in mind as I may not be able to respond if you have issues.  In other words, get started early!

\vskip.15in
\noindent\textbf{Course Objectives/Learning Outcomes:}
ABET Student Outcomes: The following student outcomes shall be supported by this coursework:
\begin{enumerate}
\item An ability to apply knowledge of computing and mathematics appropriate to the discipline.
\item An ability to analyze a problem, and identify and define the computing requirements appropriate to its solution.
\item An ability to design, implement, and evaluate a computer-based system, process, component, or program to meet desired needs.
\item An ability to communicate effectively with a range of audiences.
\item An ability to use current techniques, skills, and tools necessary for computing practice.
\item An ability to apply mathematical foundations, algorithmic principles, and computer science theory in the modeling and design of computer-based systems in a way that demonstrates comprehension of the tradeoffs involved in design choices.
\item An ability to apply design and development principles in the construction of software systems of varying complexity.
\end{enumerate}



\vskip.15in
\noindent\textbf{Student Representatives}:
These are students who are designated by letter to represent the University on official business, e.g., athletic, music, and similar events. If officially scheduled absences cause these students to miss tests, assignments, and/or other similar academic activities, University policy allows these to be made up without penalty. In accordance with this policy, Student Representatives may opt to either make up tests prior to departure, or supplanting missed tests with the final exam grade. Final exams must always be taken prior to departure to avoid an Incomplete for the course. Scheduled assignments remain subject to the lateness policy and must be turned in before departure to avoid lateness penalties. Student Representatives are responsible to inform the instructor of official absences and to make all appropriate arrangements.


\vskip.15in
\noindent\textbf{Academic Integrity}:  
Charleston Southern University abides by both Undergraduate and Graduate Academic Integrity policies.  Refer to CSU Student Handbook regarding Guidelines for the Research Paper, A Community of Honor, and the Academic Integrity Policy.  Students will have a right to appeal any removal from the program but will follow the policy provided in the student handbook related to the appeal processes. 

``Academic Dishonesty'' is the transfer, receipt, or use of academic information, or the attempted transfer, receipt, or use of academic information in a manner not authorized by the instructor or by university rules. It includes, but is not limited to, cheating and plagiarism as well as aiding or encouraging another to commit academic dishonesty.
 
``Cheating'' is defined as wrongfully giving, taking, or presenting any information or material borrowed from another source, including the Internet, by a student with the intent of aiding himself/herself or another on academic work. This includes, but is not limited to a test, examination, presentation, experiment or any written assignment, which is considered in any way in the determination of the final grade.
 
``Plagiarism'' is the taking or attempted taking of an idea, a writing, a graphic, music composition, art or datum of another without giving proper credit and presenting or attempting to present it as one’s own. It is also taking written materials of one’s own that have been used for a previous course assignment and using it without reference to it in its original form.

Students are encouraged to ask their instructor(s) for clarification regarding their academic dishonesty standards. Instructors are encouraged to include academic dishonesty/integrity standards on their course syllabi.
 
Violations of this policy will result in academic discipline, up to and including expulsion from the University.
 
For more information on procedures and violation appeals, refer to the Student Handbook.

\vskip.15in
\noindent\textbf{Disability Services}: 
If there is any student in this class who thinks he/she may have need of accommodations, that student should review the requirements/procedures on Disability Services' website \url{http://www.csuniv.edu/student-success/disabilityservices.php}. Once a student has been approved to receive accommodations through Disability Services, the student will need to contact the instructor.

\vskip.15in
\noindent\textbf{Nondiscrimination Policy and Student Rights}
Charleston Southern University does not illegally discriminate on the basis of race, color, national or ethnic origin, sex, disability, age, religion, genetic information, veteran or military status, or any other basis. Inquiries regarding the non-discrimination policies should be directed to Latitia R. Adams, Title IX Coordinator, 843-863-7374, ladams@csuniv.edu. Students should refer to the CSU Student Handbook (\url{http://www.csuniv.edu/docs/studenthandbook.pdf}) to be fully informed of their rights and remedies.

\vskip.15in
\noindent\textbf{Evaluations}: 
n order to pursue our mission of ‘Academic Excellence in a Christian Environment’, it is important that we receive feedback from students to let us know how are doing. In order to save time and paper this process is online, and should be available sometime in the second half of the semester. Students are strongly encouraged to complete the short evaluation, which is entirely anonymous. Your professor will let you know when this is active, and you can then access it through your MyCSU account. We greatly value your opinion!


\vskip.15in
\noindent\textbf{Extra Help}: 
 Dot not hesitate to come to my office during office hours or by appointment to discuss a homework problem or any aspect of the course.

\vskip.15in
\noindent\textbf{Disability Services}: If there is any student in this class who thinks they may have need of accommodations, they should review the requirements/procedures on Disability Services’ website http://www.csuniv.edu/student-success/disabilityservices.html. Once a student has been approved to receive accommodations through Disability Services, they will need to contact this instructor.

\vskip.15in
\noindent\textbf{Additional Information}

Since this is a difficult class, do not allow yourself to fall behind. There is a significant snowball potential. This class requires significantly more work than previous classes. Expect to put many (10, 20, or 30 depending on your skill) hours into the projects. Do \textbf{not} wait until the last moment. \textbf{Do} come to me for help in office hours. \textbf{Do} discuss your designs and ideas with me and not just your programming bugs.

\vskip.15in
\noindent\textbf{Important Dates and course outline}:
\begin{center} \begin{minipage}{5in}
\begin{flushleft}
Drop/Add Deadline \dotfill August 26, 1700\\
Withdraw with 'W' \dotfill October 16, 1700 \\
Course Final \dotfill November 23, 0800-1000\\
\end{flushleft}
\end{minipage}
\end{center}

\end{document}
